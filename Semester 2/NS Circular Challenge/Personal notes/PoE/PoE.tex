\documentclass[11pt]{article}
\usepackage[utf8]{inputenc}  % gebruik de juiste 'character encoding'
\usepackage[english]{babel}    % definitie van de taal (Engels is de standaard)
\usepackage{hyperref}        % geef URLs netjes weer
\usepackage{graphicx} 		 % Invoegen van plaatjes , ref: https://nl.sharelatex.com/learn/Inserting_Images
\usepackage{float} 			% MIJN VERSLAG
\usepackage{pdfpages}
\usepackage{listings}
\usepackage{multirow}
\usepackage{fancyhdr}
\usepackage{colortbl}
\definecolor{gray}{rgb}{0.6602,0.6602,0.6602}
\definecolor{lightgray}{rgb}{0.8,0.8,0.8}

\fancyhf{}

\rfoot{Page \thepage} 
\lfoot{Timothy van der Steenhoven, 522397}

\hypersetup{colorlinks,linkcolor=black,urlcolor=black,citecolor=black}



%Pagina stijl en bronmap
\graphicspath{ {image/} }   % zet het pad voor de plaatjes
\pagestyle{fancy}           % zet alleen paginanummering aan\\

\title{\textbf{Personal Profile of Excellence, Semester 2}}
\author{}
\date{\date{14 Juni 2019}}

\setlength\parindent{0pt} 	% geen irritant indents bij de paragraph

\begin{document}
	
	\maketitle

	\begin{table}[H]
		\centering
		\begin{tabular}{l c l}
		Name	& : & Timothy van der Steenhoven\\ 
		Studentnr & : & 522397\\
		E-mail &:& 522397@student.inholland.nl\\
		Study &:&Applied Computer Science \\ 
		Date &: & \today\\
		\end{tabular}
	\end{table}

\section*{Introduction}

This Profile of Excellence contains a description of the important hurdles that I encountered with my group, group 3 of de Alliantie, in the NS Circular Challenge.\\


My group consisted of Saskia van der Velden, Emmanuel Bakare and me. \\


In the beginning of this semester, I took a minor in Enschede at the Saxion University of Applied Sciences.

\section{Coaching Session 24-04-2020}

In comparison with my teammates, I took the lead in the beginning of the project. In the coaching session I was reminded that this is a \underline{group effort}. I tend to take more work on my shoulders when I sense a sign of failure or when our group does not meet my (high) expectations. \\

I had to shift my views to that of my teammates, what is it like to be in their footsteps? 
I know for an instance that Saskia was trying her best by juggling her internship with the Honours programme.\\

Emmanuel had a tough time during the corona pandemic, because his own family is far away and he struggled with his education as a result of that. I think that after I gave that thought a few days to settle, that I couldn't blame Emmanuel for not engaging as much as I would have like to see him do.\\

I empathized with his situation and am proud to have worked with him on this project, because he could do something that Saskia and I can not do at all: tell the story of our product with heart.\\

\section{Conclusion}

During this semester I was confronted with my own high expectations in the group project. I like to make the most of the project, because I like working with different people towards the same goal. \\

Therefore I had to halt myself and take a look at the situation of my teammates. What are they going through? I have 'plenty' of time to schedule for the Honours Programma, but what about my teammates.\\

At the end of the project I more frequently asked Saskia and Emmanuel how much time they could spend on the project and accepted the time they scheduled for this project, because I have thought about their situation. \\

I take this lesson with me on further projects. The ability to shift your perspective to that of a group member, consumer or manager helps with the communication between each other.

\begin{figure}[H]
	\centering
	\includegraphics[width=\textwidth]{naamloos.png}
\end{figure}






	
\end{document}